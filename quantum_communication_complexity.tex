\documentclass[a4paper]{article}
\usepackage[affil-it]{authblk}

\usepackage{hyperref}

\begin{document}
    \title{One Way Quantum Communication Complexity}
    \author{Dominic Moylett\thanks{\texttt{\href{dominic.moylett@bristol.ac.uk}{dominic.moylett@bristol.ac.uk}}}}
    \affil{Centre for Quantum Engineering,\\University of Bristol}
    \date{\today}
    \maketitle

    \begin{abstract}
        Communication complexity is the study of how much information two or more parties need to share with each other in order to perform joint computation of a problem. There are many benefits to this model of computation, from many lower bound proofs to applications in cryptography and data streaming. In this review article, we will summarise the most exciting recent trends in the one-way form of quantum communication complexity, look at what quantum speedups already exist and what problems are still open.
    \end{abstract}
    
    \section{Introduction}
    
    \section{Communication Complexity}
    
        \subsection{One-Way Communication Complexity}
        
        \subsection{Holevo's Theorem}

    \section{Functions}
    
        \subsection{Distributed Deutsch-Jozsa}
        
        \subsection{Subgroup Membership}
        
        \subsection{Permutation Invariance}
        
    \section{Relations}
    
        \subsection{Hidden Matching}
        
        \subsection{$\alpha$-matching}
        
    \section{Applications}
    
    \section{Implementations}
    
    \section{Conclusion}
    
        \subsection{Open Problems}
        
        \subsection{Other Areas}
\end{document}